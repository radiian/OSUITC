
\documentclass[12pt]{article}

\usepackage{hyperref}
\title{
The Constitution of the Oregon State University Information Technology Club
\\
(Associated Students of Information Technology Corvallis)
}

\newcommand{\article}[1]{
\setcounter{section}{0}
\setcounter{subsection}{0}

	\section*{#1}
	\addcontentsline{toc}{section}{#1}
	
}

\newcommand{\sect}[1]{
	\subsection*{#1}
	\addcontentsline{toc}{subsection}{#1}
}

\renewcommand{\thesection}{}
\renewcommand{\thesubsection}{\arabic{subsection}}
\newcommand{\revision}{ % Use this command to place the document revision at various places in the document
\begin{center}
Revision 1.0
\end{center}
}


\begin{document}
\pagenumbering{gobble}
\maketitle{}
%\revision{}

It is the purpose of this document to define the organization that is the Oregon State University Information Technology Club, herein referred to as OSUITC, and to define it's structure and operation in order to meet the goals described in this document.

\clearpage{}

\revision{}

\clearpage{}

\pagenumbering{roman}
\tableofcontents{}

\clearpage{}

\pagenumbering{arabic}
\article{Article I - Organization Name}
\paragraph{}
For legal purposes this organization shall be named Associated Students of Information Technology Corvallis. It may be abbreviated only as ASITC.  For inter university purposes this organization shall be known as the Oregon State University Information Technology Club. It may be abbreviated as the OSU IT Club or OSUITC or ITC. 

\article{Article II - Purpose}
\paragraph{}
It is the purpose of the OSUITC to facilitate the learning and exploration of all topics related to the industry of information technology. It is the purpose of this organization that in doing so the members of this organization may provide support and information to their fellow students and fellow student organizations. 

\article{Article III - Members and Membership}
\paragraph{}
Student organizations are not only operated by the students, but in most cases are also defined by it's members. In order to maintain a positive experience and a good standing with the students and the school the following sections have been declared about membership.

\subsection{Eligibility}
%\addcontentsline{toc}{subsection}{Eligibility}
\paragraph{}
All students currently enrolled at Oregon State University are eligible to join the OSUITC. Special cases may be made to allow persons not currently enrolled at Oregon State University to join the OSUITC. Membership can be achieved through the Student Involvement and Leadership club database. A link to join the club may also be published to facilitate an easier process for joining.

\paragraph{}
The OSUITC is dedicated to maintaining a non discrimination profile regardless of OSU policy. For this reason the OSUITC will not deny membership to any student on the basis of age, color, disability, gender identity or expression, genetic information, marital status, nation origin, race, religion, creed,  sex, sexual orientation, or veteran status. 

\subsection{Member Classes}
\paragraph{}
Membership is divided into Members, Non Student Members, and Privileged Members. In order to be a Member the person must be a current student of Oregon State University with a valid student ID number.
If the person is not a current student of Oregon State University then they must be a Non Student Member and may never hold any office or vote in any club matters.
Privileged Members are those who have paid a due for a specific privilege. The privileges associated with the dues are to be determined by the club leadership along with the dues to be paid for the privilege.
Non Student Members may become Privileged Members and enjoy all privileges associated with the class
of member. As such the privileges associated with the Privileged Member class may never include 
the right to hold an office or the right to vote in any matter.

%esubsubsection{Non Member}
%\addcontentsline{toc}{subsubsection}{Non Member}
%\paragraph{}
%A Non Member is anyone who is not registered as any class of member within the organization. Certain club events shall be open to Non Members in order to allow all students a chance to p%articipate in club activities before commiting to joining the OSUITC.

%\subsubsection{Member}
%\addcontentsline{toc}{subsubsection}{Member}
%\paragraph{}
%A Member is the most basic class of membership. No payment, fee, or service requirement shall be issued for any student to become a Member. All Members are granted one vote in all votes not restricted to executive members. Members are eligible to participate in all club events. Members may be restricted in which club owned equipment they are able to use. A current and valid OSU ID card and number is rquired to become a member.

%\subsubsection{Advanced Member}
%\addcontentsline{toc}{subsubsection}{Advanced Member}
%\paragraph{}
%Advanced Members share the same rights as Members with the sole exception of the ability to access and use club owned equipment. This use includes both physical access to equipment as well as virtual access to club equipment and resources.

%\subsubsection{Non Student Member}
%\addcontentsline{toc}{subsubsection}{Non Student Member}
%\paragraph{}
%The class of Non Student Member is reserved for special cases where a person who is not currently enrolled at OSU has been granted membership. This class of member does not share the same rights as the other members. Non Student Members may present no votes in any club matters. Non Student Members may not hold any office within the organization as per OSU regulations. Non Student Members may participate in all other club events.

%\subsubsection{Executive Member}
%\addcontentsline{toc}{subsubsection}{Executive Member}
%\paragraph{}
%The class of Executive Member is granted to all members that hold an office in the organization. The class is granted on the election or appointment of the member to an office, and revoked on the absence or removal of the member from an organization office. Executive Members hold the rights of Members with a few additions. Executive Members shall have far less restrictions on access to club owned equipment and resources. Executive Members are also granted one vote in all executive votes.




%
%Combine the next two sections into one
%
\subsection{Member Dues}
At the end of each year the club leadership may determine the dues required for membership for the following year.
This includes dues for basic membership as well as dues for privileged membership.



\subsection{Member Requirements}
%\addcontentsline{tox}{subsection}{Member Requirements}
\paragraph{}
In order for a member of any class to remain current with the club they must be present at each member
meeting. If the member is not present at a member meeting their membership for the following term 
will not be renewed automatically but they will remain in good standing with the club and be 
eligible to renew their membership.

\paragraph{}
For a member to gain a vote in any business voted on at a member meeting the member must have been 
present at the member meeting for the term previous excepting Summer term. For a member to gain a 
vote for a Fall term member meeting the member must have been present at the previous spring member
meeting. 

\paragraph{}
If a member knows prior to a meeting that they will be unable to attend, that member may make 
arrangements with an officer to receive a vote in the next term's member meeting.

\paragraph{}
No member shall ever be required to hold any obligations to the club in order to retain membership.
As such no member may ever be required to hold any office or responsibilities in order to remain
a member in good standing.

\subsection{Disciplinary Action}
\paragraph{}
At times disciplinary action may be required for members of all classes for violating the constitution, mistreatment of other members or club property, general immoral conduct, violation of OSU policies,
or violation of local or federal law. 
Any member accused of such an offense has the right to a trial by their peers. 
Should the member admit to guilt before an Officer of the club the Member waves their right to a trial and the executive board shall collectively decide the actions that need to be taken by a majority vote.

\paragraph{}
Should the decision need to be determined by the executive board, a meeting shall be called at a date such that all officers and the guilty party may be present.

\paragraph{} 
All trials must be presided over by at least one Officer of the club and a jury of no more than 10 and no less than 5 non executive members will be constructed by a random drawing of names. 
Any member who's name is drawn for a jury must accept the summon unless it poses an extreme burden on the summoned. In order to lessen the burden of a summons a minimum of 10 days notice must be given before a trial. The time and place of the trial must be set upon so that all members called to the jury and a presiding Officer may be present. 

\paragraph{}
In order for a jury's decision to be upheld all jurors must agree unanimously. In extreme cases the accused may petition the jury's decision.  
\\
Disciplinary action may take any or all of the forms listed here:
\begin{itemize}
\item Probation from club and club events
\item Probation from using club equipment
\item Suspension from club and club events
\item Suspension from using club equipment
\item Revoking of advanced member class
\item Removal from office
\item Barring from holding any present or future office
\item Removal from the club
\end{itemize}

\subsection{Resignation}
\paragraph{}
Membership of this club has no bindings and as such a member may resign from the club at any time regardless of any duties they may hold to the club. Members not holding office may resign at any time by contacting an administrator of the club. Resignation carries no penalties and as such any member who has resigned may join the club again at any point. In order for an Officer to honorably resign they must submit notice of resignation two weeks prior to vacating the held office. Upon return to the club they may then campaign to hold another office if they desire. If they do not submit a notice of resignation two weeks prior to vacating the office they will lose their eligibility to ever hold any office in the club. 

\paragraph{}
However at some point an exceptional case may arise in which an officer is unable to give notice
of resignation for some extreme circumstance. If an officer must resign unexpectedly and immediately
they may make arrangements with the President to resign honorably under extreme circumstances.

\paragraph{}
If a member has paid any dues or fees and resigns from the club there will be no refunds to the member unless not issuing a refund will cause an excessive burden on the member. Refunds must be appealed to and approved by both the Treasurer and the Vice President.


\article{Article IV - Officers}
\paragraph{}
It is necessary for the well being and the fluid operation of this club to appoint several officers to administrate and manage the club affairs.

\subsection{The Officers}
\paragraph{}
The officers shall be one President, one Vice President, one Secretary, one Treasurer, one Director of Operations, and two Representatives. These Officers shall be elected to uphold and enforce this Constitution and the Bylaws of this organization. They are to do it to the best of their abilities and attempting to enforce them as they were intended to be enforced. The duties of each office shall be outlined in a subsequent article. All duties not explicitly written in the subsequent article for the office shall not be required of the officer to maintain office.

\paragraph{}
In order for the club to function properly the Office of The President must be filled at all times.
All other offices may remain vacant unless otherwise described in this document. 
In the event that the Office of the President becomes vacant
the next office on the following list shall assume the presidency. 

\begin{enumerate}
\item The President
\item The Vice President
\item The Director of Operations
\item The Secretary
\item The Treasurer
\item A Representative
\end{enumerate}


\paragraph{}
If a representative is required to assume the office and both seats for the representatives are 
filled then the two representatives shall play rock paper scissors for the office, best two of three
with no redos.

 
\subsection{Nominations and Elections}
\paragraph{}
In order to facilitate a seamless exchange of officers it is necessary to hold elections such that the officers elected may better represent the majority of the clubs needs. The requirements and process of this exchange is outlined in the following sections.

\subsubsection{Election Timeline}
\paragraph{}
Elections must happen within one school year. This is so that any graduating members that hold office will have a replacement for the next school year. Applications for office will open at the 
beginning of every year with the selected applicant taking office at the beginning of the following 
year. Applications will close at the end of week 1 of spring term. Voting will then take place at the Spring term member meeting every year.

\subsubsection{Applications}
\paragraph{}
Any member who wishes to run for an office must submit an application before Monday of week 2 of spring term. 
This is so all applications can be verified before voting happens at the spring members meeting.
Applicants must have nominations from at least two other members not holding an office in the club. 
Applications must be verified by an officer to ensure validity and that the member meets the eligibility requirements for the office they are running for.
It is the duty of the applicants to promote themselves for the position they are applying for. 

\subsubsection{Election and Votes}
\paragraph{}
Voting will take place at the spring term members meeting. 
It shall be done electronically to ensure a prompt decision and accurate results. 
The candidacy is won by a plurality.
If desired by the members a revote can be called if two fifths of the student members of the club call for a revote. 
The revote shall be done in the same manner as the original vote and must happen immediately after
the original vote.
If a revote is called the second decision is final an no more votes may be called. 

\subsubsection{A Tie Vote Between Two Candidates}
\paragraph{}
At some point in the life of the club there may be a tie between two candidates. In order to resolve a tie vote the following procedures must be adhered to strictly. First the two candidates must Rock Paper Scissor to pick even or odd numbers. The winner of the best two out of three Rock Paper Scissor may choose between even or odd numbers. Then a twenty sided die will be rolled by the current club president on a large open surface. If the die lands on the set that was picked then that candidate wins the election. If the die lands on the set that was not chosen then the other candidate wins the election.

\subsubsection {A Tie Vote Between Three or Four Candidates}
\paragraph{}
In the unlikely event of a three way or four way tie, the procedures are similar to that of a standard tie. Rock Paper Scissor must be played tournament style to determine first, second, and third place winners. This determines the order in which the candidates will choose a range of numbers on a twelve sided die. If it is a three way tie the candidates will choose from the ranges 1-4, 5-8, and 9-12. If it is a four way tie the ranges of numbers to be chosen from are 1-3, 4-6, 7-9, and 10-12. After the ranges have been chosen the current president of the club will roll a 12 sided die on a large open surface. The range that the number lands in determines the winner of the election.

\subsubsection{Ties between Five or More Candidates}
\paragraph{}
Should there arise a tie between five or more candidates an immediate revote must be called, even if a revote was already called. 
If the tie persists then the candidacy shall be determined by a trial of open combat between the candidates. 
Each candidate will receive a nerf blaster and a pair of safety glasses. 
Candidates are not allowed to touch each other in any way other than by tagging with a nerf dart.
If a candidate is tagged they are out of the election. The last candidate standing shall have the office.

%\subsection{Eligibility}
%\paragraph{}
%In order to maintain the integrity of this government several eligibility restrictions shall be made for the various offices. Members may only run for one office at a time. The only way for a member
%to hold multiple offices is if they are appointed to that office by the president as deemed necessary in certain situations such as an unexpected office vacancy.

\subsection{The Office of the President}
\subsubsection{Purpose and Duties}
\paragraph{}
It is the purpose of the President to oversee the club in it's entirety. To guide the club and other officers towards the goals of the club as well as to oversee in setting those goals. The President shall
also be in charge of all decision making unless a process for a decision is specifically outlined in the constitution. The president may also delegate some decisions to other officers unless
restricted by the constitution.

\subsubsection{Eligibility}
\paragraph{}
In order for a Member to be eligible to fill the office of the President there are requirements that must be met. First and foremost the office of president cannot be held by any Non-Member or Non-Student Member as required by OSU policy for student organizations. Given that the Member is not of the previously mentioned classes the member must also have maintained the status of Member for no less than one year. 

\subsection{The Office of the Vice President}
\subsubsection{Purpose and Duties}
\paragraph{}
It is the purpose of the Vice President to assist the President. In assisting the President, the Vice President may be delegated tasks by the President. The Vice President also exists to serve as the 
President in circumstances when the President may be unable to perform their duties to the club.
\subsubsection{Eligibility}
\paragraph{}
Eligibility for the Office of the Vice President shall be the same as those of the Office of the President.

\subsection{The Office of the Secretary}
\subsubsection{Purpose and Duties}
\paragraph{}
It is the purpose of the Secretary to serve both the President, the other officers, as well as the club itself. The Secretary exists to handle the logistics of event planning. The Secretary also exists
to take the minutes at all official club meetings as necessary as well as maintain accurate records of critical information regarding and serving the club.
\subsubsection{Eligibility}
\paragraph{}
In order to be eligible for the Office of the Secretary the person must be both a member of the club and a student of Oregon State University as per university regulations regarding student organizations.

\subsection{The Office of the Treasurer}
\subsubsection{Purpose and Duties}
\paragraph{}
It is the purpose of the Treasurer to oversee and maintain the finances of the club. The Treasurer exists to keep accurate and current records regarding club funds as well as expenditures from the treasury.
The Treasurer should also be able to provide and assist in financial planning and budgeting for the club and any club needs regarding funds.
\subsubsection{Eligibility}
\paragraph{}
In order to be eligible for the Office of the Treasurer the person must be both a member of the club and a student of Oregon State University as per university regulations regarding student organizations.
The Office of the Treasurer may also be subject to a background investigation as the officer will be responsible for handling and managing money.
It is up to the President to asses the results of any background investigations and decide if the 
results of the background investigation should bar the candidate from holding office.


\subsection{The Office of the Director of Operations}
\subsubsection{Purpose and Duties}
\paragraph{}
It is the purpose of the Director of Operations to oversee and handle relations regarding club members and other clubs. The Director shall also be responsible for overseeing proper use and retention of club property as 
part of member relations to the club. 
\subsubsection{Eligibility}
\paragraph{}
In order to be eligible for the Office of the Director Of Operations the person must be both a member of the club and a student of Oregon State University as per university regulations regarding student organizations.
The Director of Operations should also be able to maintain a positive outlook and excellent social
skills for keeping good relations between members, other organizations, and outside associations. 


\subsection{The Office of the Representative}
\subsubsection{Purpose and Duties}
\paragraph{}
It is the purpose of the Representative to assist the Director of Operations in maintaining their responsibilities. The Representatives shall also act as representatives of the club as needed by the club. They shall show
the utmost respect to any entity they will represent the club too as well as showing great taste and class. They shall keep accurate and current knowledge of the club's functions and capabilities as well
as services offered to outside entities.
\subsubsection{Eligibility}
\paragraph{}
In order to be eligible for the Office of the Representative the person must be both a member of the club and a student of Oregon State University as per university regulations regarding student organizations.

\subsection{Vacancies}
\paragraph{}
All offices excepting the Office of The President are allowed to remain vacant. If a vacancy other 
than that of the President occurs it is up to the President to decide how to precede. The President
may ask other officers to assume some or all of the roles of a vacant position. However no officer
is obligated to take on such roles unless they are required to maintain their current office. 

\subsection{Removal From Office}
\paragraph{}
The occasion may arise when an elected officer is unable to fulfill their duties. If another officer feels that an officer is not fulfilling their duties it shall be brought to a vote at an executive meeting.
The accusing party must be able to provide sufficient evidence that the officer is not fulfilling their duties and a decision to remove the officer will be made by a majority vote of all officers excepting the accused party. 
If removed from an office, the person shall not be eligible to hold that office in the future. 

\article{Article V - Advisers}
\paragraph{}
In order to to fulfill the requirements for a sponsored student organization there must be an adviser present from each sponsoring unit. 
The adviser will serve as a resource to the club when advice and guidance in club matters are needed. 
The requirements of each adviser shall be determined during sponsorship negotiations with the school
providing sponsorship or during renewal of sponsorship agreements as necessary.

\article{Article VI - Meetings}
\paragraph{}
In order to maintain the club it may be necessary to hold frequent or infrequent meetings. This article shall cover the meetings that must happen for the club to function, as well as requirements for meetings that are not outlined in this document.

\subsection{Member Meetings}
\paragraph{}
Each term excepting Summer term there shall be an all member meeting in which members are required to attend to gain voting rights. 
These meetings must happen in week four of the term. 
The time and place of the meeting must be published by club officials by the end of week two of the given term. 
If the time and place of the meeting is not given by week two then the executive board must make 
arrangements to move the meeting to a later time giving at least 2 weeks prior notice to the meeting.
A member meeting must happen every term except summer.

\subsection{Executive Meetings}
\paragraph{}
The officers of the club must meet at least once per term excepting Summer term to discuss official club business.
It shall be suggested that any acting advisers be present at all executive meetings. 
These meetings shall be arranged in a manner such that all officers may be present.

\subsection{Disciplinary Meetings}
\paragraph{}
%+++++++++++++++++++++++++++++++++++++++++++++++++++++++++++++++
If disciplinary action is required a meeting shall be called for the proceedings of a hearing
or for the decision of disciplinary action. Arrangements for a disciplinary meeting must be
made such that all parties required to be present may be present. Notice of the meeting must
be given no less than 10 days before the meeting is held. 

\paragraph{}
Disciplinary meetings shall be closed to the public but all members may be present.

\subsection{Other Meetings}
\paragraph{}
There may arise occasions which require a club meeting that is not outlined in this document. For that case notice of the meeting must be posted no less than one week (seven days) prior to the date of meeting. The meeting must be communicated across all channels to ensure all members are aware of the meeting. Official meetings may only be called by club officers. Any member who wishes to declare an official meeting must consult an officer about it.

\article{Article VII - Quorums}
\paragraph{}
In order to maintain the integrity of the club several quorums shall be set to ensure that votes
accurately represent the club.

\subsection{Member Meetings}
\paragraph{}
In order for a vote taken at a member meeting to be valid 40 percent of the members must be present.

\subsection{Amendment Votes}
\paragraph{}
In order for an amendment to be voted on 60 percent of the members must be present.

\article{Article VIII - Amendments}
\paragraph{}
As the club grows it may be necessary to amend this Constitution. Amendments must be formally 
submitted in writing at a Member Meeting where the amendment is to be read aloud. 
As all Constitution amendments must be added by the end
of the year the amendment must be voted on at or before the Spring term member meeting. If an
in progress amendment has not been voted on after the Spring term member meeting it will roll
over to the next school year where it may again be voted on. Amendments may not be voted on at
the same meeting in which they were submitted.

\paragraph{}
All proposed amendments must be made available to view to all members of the club so that they
may decide how they want to vote on it.

\paragraph{}
If an amendment is voted against then the amendment may either be withdrawn by the submitter or
revised and resubmitted. 

\paragraph{}
An amendment is passed if 75 percent of all members present during voting including officers vote for the amendment. 




\article{Article IX - Parliamentary Authority }
\paragraph{}
The Parliamentary Authority shall be Robert's Rules Of Order.


\end{document}
