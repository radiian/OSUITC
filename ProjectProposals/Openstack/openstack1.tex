\documentclass[12pt]{article}
\usepackage[margin=1.0in]{geometry}

\title{
\textbf{Hardware Usage Proposal}
\\
\textit{Information Technology Club, OSU}
}
\author{Zachery Thompson, President}
\begin{document}
\pagenumbering{gobble}


\maketitle{}
\clearpage{}

\tableofcontents{}
\clearpage{}


\pagenumbering{arabic}
\section{Proposal Purpose}
This document serves to outline the proposed use of resources that have been granted to the Information Technology Club.
The intent of this proposal is to create a framework and a set of guidelines for how new resources can be used to bring
club members together for new learning opportunities and a chance to gain hands on experience. It will cover the end
goal for the project as well as some steps to get there. 

\section {Project Purpose}
This project has multiple purposes for the IT Club. The primary purpose of this project is to provide club members
hands on experience with multiple areas of the IT field. Members will have access to hands on experience in networking,
hardware deployment, OS and Hypervisor installation, virtual machine setup and management, storage system setup and management,
general system administration, and more. The goal is to create a project that is large enough that one individual would not
be sufficient enough to manage alone, encouraging others to participate and take ownership of certain aspects of the project.
\\
The secondary purpose of this project is to provide stable computing resources to club members. Once this project is complete 
the IT Club will have a functioning virtual machine hosting environment with many features. The goal is that this new system
will replace the current OpenStack system and become superior to it.

\section {Hardware Overview}
Three servers are being granted to the IT Club for use in this project. They are being provided by the Server Infrastructure Group
at Information Services for use by the IT Club. The three new systems are Dell R610 servers each with dual processors, memory, and
hard drives already installed. They will be installed with a Dell R510 that the IT Club already owns. A Cisco Catalyst 3750G 48
port network switch will handle network connectivity to the hosts and a white-box pfSense router will handle firewall and routing
for the cluster. All together the cluster should occupy 6 rack units of space. 

\section {Software Overview}
The bulk of this project will be in installing software and managing it. The project will use a wide variety of software and 
operating systems to build up the infrastructure required to run OpenStack. Windows Server 2016 will be used for Active Directory
and user management, and VMWare ESXi will be used for the computing hosts to make better use of resources. 

\subsection{Windows Server 2016}
Windows Server is the OS of choice for user rights management across the entire cluster. Active Directory is widely used in many
large and small organizations to maintain users and groups and grant access to various systems and services. Using Windows Server
will not only provide a robust administration environment, but also expose members to systems and processes in use in nearly every
business. The Active Directory environment will be installed on the Dell R510.

\subsection{VMWare ESXi}
ESXi (or vSphere) is a widely used enterprise grade hypervisor. It allows a single host, or cluster of hosts, to be provisioned with
multiple tenant operating systems to divide resources between different individuals or services. ESXi will be the base layer for all
other services for the IT Club. Other services will reside within virtual machines on the ESXi cluster. It is easy to install and manage
and will provide members hands on access to the industry leading virtualization platform.

\subsection{Other Services}
The few remaining services that will be implemented in this project will reside within virtual machines on the ESXi cluster. They will 
provide all the facilities required to run OpenStack effectively. This will include a number of virtual machines providing the control
plane for OpenStack, computing resources for OpenStack, and a Ceph storage cluster for distributed file system services for OpenStack. 

\subsubsection{OpenStack Services}
The OpenStack system will be implemented on separate VMs distributed across the cluster. The suggested OS for this is CentOS 7. CentOS
is a commonly used enterprise grade Linux distribution, however other distributions will support OpenStack and may be used at the
request of the club members deploying the system. 

\subsubsection{Ceph FS}
Ceph FS is a distributed file system capable of using off the shelf hardware to provide distributed storage capabilities will low effort.
It is a redundant and resilient system capable of scaling easily. Ceph FS will provide storage for end user VMs running on OpenStack. 
It will allow OpenStack to migrate VMs between compute hosts in the even of a node failure. It will also allow for much more flexible
use of storage resources.

\section{Implementation}
The entirety of this project is designed to be implemented by the club members. Everything from design and planning to rack and install
is to be done by the members of the IT Club. Member meetings will take place to finalize designs and plans before installation will take 
place. This will leave the plan open to peer review and suggestions as the systems come together. This will also present opportunities to
new members to get involved, or current members to get involved in different parts of the project. Installation will be split between on
premise work and remote work, with the majority of the work taking place remotely once the base infrastructure is in place. If the system
is implemented properly then the majority of management tasks will be accomplished remotely over the network. Special care must be taken
during planning to ensure that all necessary management facilities are accessible over the network.  


\section{End Results}
Once the project is completed the IT Club will have a well rounded set of resources for hosting virtual machines. This system can be used
to provide resources to club members for their own experiments and also provide resources for the IT Club to host for members. However if
the club so chooses, the whole thing can be torn down and rebuilt at a later date as an exercise in repetition. This would provide new members
a chance to participate if they were not able to previously, and other members who did participate previously to take part in a different 
area of the project. The club may also find that OpenStack is not suitable for providing resources when implemented as planned and a new plan
can be developed to provide resources to members. The end result will be the same regardless of what happens at the end. Opportunities to learn
will have been created and fulfilled and the quest for knowledge furthered. Club members will have been able to find new interests or disinterests,
and new areas of IT in which to study. Members will have formed new relationships and bonds between other members and the IT Club will grow stronger.
 
\end{document}
