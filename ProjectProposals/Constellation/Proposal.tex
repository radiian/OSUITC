\documentclass[12pt]{article}
\title{
Project Constellation Proposal
\\
Information Technology Club, OSU
}
\author{Zachery Thompson}
\begin{document}

\maketitle{}

\clearpage{}

In the last 10 years or so the internet has grown exponentially and begun to invade our every day lives. 
We rely on it for entertainment, news, and even for our business affairs. Many businesses
rely heavily on the internet for their inner workings, and even some businesses make their money by providing
and facilitating connections to the wider internet. For the vast majority of users on the world wide web, 
knowing how the inner works of the internet happen are simply not necessary and maybe even above their
knowledge level. But for many engineers working with all forms of emerging technologies, having a better
understanding of the inner workings of the world wide web is either a necessity, or would be a significant help
to their work.  
\\

The concepts and technologies employed on the world wide web can be taught in classrooms and even explored in
labs in a very basic sense. But to actually see the world wide web in action would require the observer to
be an active part in the network. This is an incredibly daunting challenge for anyone acting on their own 
behalf. Therefore it is very difficult to study how the world wide web functions in it's entirety.

\clearpage{}

\section{Project Constellation Overview}

Project Constellation aims to emulate the world wide web on a much smaller scale and in a more controlled
environment. The creation of such a network for academics would allow students to directly interact with a 
network very similar to the world wide web in order to study the technologies and processes that create the
larger world wide web.
\\

While the world wide web consists of thousands of very expensive routers all interconnected, such a network
can be emulated using inexpensive off-the-shelf hardware that could be supplied by each participant of the
project.
\\

Through collaboration on the part of all participants a large network acting very closely to the world wide 
web can be achieved. Such a network could provide a complete platform to allow further exploration of the topics
involved with the world wide web such as: DNS, Various routing protocols, multi-homed networks, and many others.


\section{Materials and Methods}

There are many different systems and methods that can be used to interconnect routing devices on the internet. 
However the primary protocol used to stitch together the internet appears to be Border Gateway Protocol (BGP).
For this project we will focus on using BGP as the core networking protocol, but many other protocols can be 
used for edge routers or specific connections between peers for the purpose of learning.


\subsection{Border Gateway Protocol}

The protocol known as Border Gateway Protocol (BGP) is a very powerful yet easy to employ routing protocol that
is widely used between routers on the world wide web. It uses the concept of peering to connect routers as neighbors.
In this concept each neighbor may have many other neighbors, and each neighbor is communicates the routes on the network
it has learned to all of its other neighbors. This allows a router on one side of the world to propagate it's advertised
addresses across the entire internet.
\\

Entities acting on the internet are referred to as Autonomous Systems (AS). An AS encapsulates a single entity on the network
and may be a single router connected to the internet, or a small interconnected cluster of routers each connected to the internet
in different locations. In order to keep track of all this, each AS needs a special and unique number (an ASN) so that all systems on the network
can differentiate other systems. Multiple interconnected routers owned by the same entity can employ the same AS number to create one
large AS instead of using a single system per ASN.
\\

When connecting (called peering) two routers via the BGP protocol, each router must be assigned an ASN. The two routers then must be connected via 
some ethernet capable medium and each interface in that connection must exist within the same subnet as the other router. Then each router must
be configured to add the other as a neighbor. Once both routers have been configured with each other as a neighbor the BGP connection has been
completed and the two routers can now share their routes. BGP is a very easy to implement protocol but carries significant power in the routing
world.
\\

This project will not be limited to just BGP and may incorporate connections using RIP, OSPF, and even static routes. However the primary transport
layer of this project should use BGP.


\subsection{VyOS}

In order to use off-the-shelf hardware for this project a routing operating system must be chosen for each node on the network. Ultimately the routing
platform choice will be up to the person or entity implementing the router on the network. There are many choices that can support BGP, some do so
out of the box, and others require additional software updates to support BGP.
\\

The officially suggested routing platform for this project is VyOS. VyOS is a fork of the Vyatta project which uses the Quagga routing engine. VyOS
is capable of implementing routes via all of the major routing protocols out of the box from installation. It is light weight and capable of running
on minimal hardware. It has a powerful command line interface that is loosely based on JunOS from Juniper. The command line syntax is easy to pick up
and learn quickly. It can also easily be virtualized for easy testing and simulation of a full network from a single virtualization host. 
\\

As VyOS is so lightweight and simple to implement it will allow participants to select cheap hardware for their physical routing platform. 


\subsection{Physical Hardware}

Physical hardware selection will ultimately be up to the person or entity implementing the network segment just as with the routing OS. However
do to physical and power constraints of physical placement of such devices there are a few guidelines that should be considered for physical hardware.
\\

First and foremost the hardware selected should be able to effectively run the selected routing OS. It should be able to successfully perform it's routing
and switching duties as well as be able to hold the routing table of the entire network. The host should also be power friendly. All hosts should not consume
more than 500W of power. The host should be quiet enough that you could comfortable take a test with the host running on the next desk over from you. The host
should also be as small as possible while still being able to meet all of the other criteria. A small footprint for a host will allow it to be placed in such
a way that it will remain out of the way of any day to day activities that may take place in the area it occupies.
\\

While those guidelines exist to ensure that no piece of this network would interfere with any on campus activities, there are other guidelines to consider
for an effective router. It is suggested that each node acting as a router have more than 2 network ports. 3 network ports would allow for each node to peer
with two other systems on the network and still provide network connectivity to some form of internal network maintained by the owner of the node. For long
distance connections to other peers a fiber optic networking card may be required and is highly suggested.
\\

Some suggested systems are:

\begin{itemize}
	\item Dell Optiplex 755 SFF
	\item Dell Optiplex 760 SFF
	\item Dell Optiplex 780 SFF
	\item Dell Optiplex 790 SFF
	\item HP Elite 8200 SFF
	\item Lenovo E71 SFF
\end{itemize}

\subsection{Physical Links}

One of the primary barriers of this project is physically connecting the routers on the network. Hosts that reside within the same room would be capable
of utilizing standard copper UTP cables to create links. However signals over UTP copper cables may begin to degrade past 100 meters (about 300 feet). 
For longer distances it will be necessary to utilize fiber optic connections. The suggest fiber cabling type for this entire project is OS2 Single Mode
fiber cables with LR optics. This will ensure that all fiber connections can travel any distance required by the network. 

\end{document}
